% This is samplepaper.tex, a sample chapter demonstrating the
% LLNCS macro package for Springer Computer Science proceedings;
% Version 2.20 of 2017/10/04
%
\documentclass[runningheads]{llncs}
%
\usepackage{graphicx}
\usepackage[brazil]{babel} % for European Portuguese use portuguese
\usepackage[fixlanguage]{babelbib}
\selectbiblanguage{brazil}

% Used for displaying a sample figure. If possible, figure files should
% be included in EPS format.
%
% If you use the hyperref package, please uncomment the following line
% to display URLs in blue roman font according to Springer's eBook style:
% \renewcommand\UrlFont{\color{blue}\rmfamily}

\begin{document}
%
\pagenumbering{gobble}
\title{Fusão de informações entre imagens PolSAR e ópticas para reconhecimento de padrões}
%
%\titlerunning{Abbreviated paper title}
% If the paper title is too long for the running head, you can set
% an abbreviated paper title here
%
\pagestyle{plain}
\author{Autor: Gabrielle Cibyl da Hora \\ 
Orientador: Mauricio Marengoni} 
%
%\authorrunning{F. Author et al.}
% First names are abbreviated in the running head.
% If there are more than two authors, 'et al.' is used.
%
\institute{PPGEEC - Programa de pós-graduação em Engenharia Elétrica e Computação. 
Universidade Presbiteriana Mackenzie.\\
\email{gcibyl@gmail.com}}
%
\maketitle              % typeset the header of the contribution
%
\begin{abstract}
% AAB modificado
O projeto de pesquisa propõe a fusão de informações para reconhecimento de padrões entre imagens ópticas e canais hh, hv e vv de imagens PolSAR múltiplas visadas. Os métodos de fusão foram escolhidos a partir das referências \cite{bmf_2020} e \cite{ref_proc4}.  Será estudado a viabilidade do métodos de fusão por média simples, transformada wavelet discreta de multi-resolução (MR-DWT), análise de componente principal (PCA), estatísticas ROC, transformada wavelet estacionária de multi-resolução (MR-SWT) e um método multi-resolução baseado na decomposição em valores singulares (MR-SVD) com a inserção das imagens ópticas. Com o objetivo de classificar de maneira acurada as bordas vamos propor novas técnica de reconhecimento de padrões. 
%A primeira técnica baseada em votação majoritária e a segunda técnica baseada em propriedades estatísticas das imagens. 

\addto\captionsenglish{\renewcommand{\keywordsname}{Palavras-chave}}
\keywords{Imagens PolSAR  \and Imagens óticas \and Reconhecimento de padrões \and \textit{Deep Learning} \and Inteligência Artificial}
\end{abstract}
%
%
%\
\section{Introdução}

O presente projeto de pesquisa será submetido ao Programa de Pós-graduação em Engenharia Elétrica e Computação da Universidade Presbiteriana Mackenzie. 

O radar de abertura sintética polarimétrico PolSAR alcançou uma posição essencial no sensoriamento remoto. Os dados que esses sensores fornecem precisam de técnicas especificamente adaptados para reconhecer as informações disponíveis. Entre essas técnicas, a detecção de borda é uma das mais importantes para extrair informações. As bordas possuem um alto nível de importância e dificuldades, como tal, fornecem informações relevantes sobre a imagem.

% AAB inserido
As pesquisas em \textit{Deep Learning} e reconhecimento de padrões são muito importantes na área de sensoriamento remoto, uma pesquisa sistemática dos métodos são encontradas na referência \cite{ztmxzxf}. 
As técnicas serão aplicadas no processo de detecção de bordas e fusão das informações de bordas. 

A abordagem de modelagem estatística, principalmente as técnicas descritas em \cite{bmf_2020}, \cite{nhfc},  e \cite{ref_proc3}, para a detecção de bordas, são usadas para fornecer informações ao processo de fusão, para isso, temos como base do trabalho a referência \cite{ref_proc4}. 

A ideia principal proposta no trabalho é detectar bordas em cada imagem (canal) de uma imagem PolSAR, ampliando o método para imagens óticas, e realizar a fusão das informações de bordas, com o objetivo de melhorar a qualidade da informação detectada.

\section{Objetivos}

O principal objetivo do projeto é realizar a fusão de informações provenientes de aplicações das técnicas de reconhecimento de padrões para detecção de bordas sem utilizar métodos que suavizem os ruídos do tipo \textit{speckle}. Para alcançar o objetivo propomos novas técnicas de fusão ampliando os resultados alcançados na referência \cite{bmf_2020}.

\section{Justificativas e motivações}

Tem sido um desafio propor, para a área de sensoriamento remoto e computação, a utilização das imagens PolSAR em aplicações devido a necessidade de estudos mais profundos. Para uma melhor compreensão e interação entre radar e objeto de estudo precisariamos de dados bem calibrados, algoritmos mais robustos e melhores técnicas de processamento de imagens, principalmente para o radar de abertura sintética polarimétrica (PolSAR) devido a existência de ruídos do tipo speckle.

Este projeto de pesquisa apresenta tópicos atuais que servem de motivação e justificativa ao projeto, por exemplo:
\begin{itemize}
\item Aplicação em estudos florestais.
\item Pesquisa mineral.
\item Sensoriamento remoto.
\item Topografia e geologia.
\item Oceanografia e glaciologia.
\item Agricultura.
\item Alvos fixos ou em movimento.
\item Monitoramento ambiental.
\item Controle de derramamento de petróleo.
\item Auxílio de sistemas óticos.
\end{itemize}

\section{Conceitos}
% AAB inserido
O projeto pode ser dividido em uma primeira etapa que consiste em encontrar informações de bordas nos diferentes canais (imagens) e uma segunda etapa que consiste na fusão das informações provenientes da primeira etapa. 
\subsection{Detecção de bordas em cada canal}
 Vamos modelar as imagens PolSAR com a função de densidade de probabilidade para os canais $(hh)$, $(hv)$ e $(vv)$ usando a distribuição Wishart (PDF) descrita por
\begin{equation}
    f_{\mathbf{Z}}(\mathbf{Z};\mathbf{\Sigma},L)=\frac{L^{mL}|\mathbf{Z}|^{L-m}}{|\mathbf{\Sigma}|^{L}\Gamma_m(L)} \exp(-L\mathbf{tr}(\mathbf{\Sigma}^{-1}\mathbf{Z})), \\
\end{equation} 
onde, $\mathbf{tr}(\cdot)$ é o operador traço de uma matriz, $\Gamma_m(L)$ é uma função Gamma multivariada definida por
\begin{equation}\label{cap_acf_10}
	\Gamma_m(L)=\pi^{\frac{1}{2}m(m-1)} \prod_{i=0}^{m-1}\Gamma(L-i) \\
\end{equation}
e $\Gamma(\cdot)$ é a função Gamma. Podemos afirmar que $\mathbf{Z}$ é distribuído como uma distribuição Wishart denotando por $\mathbf{Z}\sim W(\mathbf{\Sigma}, L)$ e satisfazendo $E[\mathbf{Z}]=\mathbf{\Sigma}$. O simbolo $\mathbf{\Sigma}$ representa a matriz de covariância associada a $\mathbf{S}$.
 
Usando a função densidade de probabilidade e o método da máxima verossimilhança vamos detectar as bordas nas imagens PolSAR. Da mesma forma temos que definir o procedimento para imagens óticas. 
\subsection{Métodos de Fusão para detectar bordas em cada canal}
 O objetivo da fusão de imagens é mesclar o conteúdo de várias imagens da mesma cena (canais diferentes) em uma única informação, assim, obtendo a melhor correção ao comparar individualmente cada imagem. Os seguinte métodos de fusão foram estudados na referência \cite{bmf_2020},
\begin{itemize}
  \item Média simples;
  \item transformada wavelet discreta de multi-resolução (MR-DWT);
  \item análise de componente principal (PCA);
  \item estatística ROC;
  \item transformada wavelet discreta de multi-resolução (MR-SWT);
  \item transformada baseada em decomposição em valores singulares de multi-resolução (MR-SVD).
\end{itemize}

A classificação é questão fundamental no
processo de interpretação de imagens PolSAR. Diversos estudos propõem desenvolvimentos de classificações de imagens PolSAR, tanto para métodos clássicos quanto híbridos. O projeto tem como intuito comparar e propor novas novas técnicas baseadas nos seguintes tópicos 
\begin{itemize}
  \item SVM, veja referência \cite{ref_proc1}.
  \item Deep Learning, veja referências \cite{ref_proc3} e \cite{ztmxzxf}.
  \item Forest Classification, veja referência \cite{ref_proc1}.
\end{itemize}

\section{Metodologia}
Descrevemos a metodologia nos seguintes itens,
\subsection{Estudo teórico}
\begin{itemize}
  \item Realização da pesquisa bibliográfica e estudo dos artigos obtidos.
  \item Estudo da viabilidade e decisões sobre abordagens que serão propostas.
\end{itemize}

\subsection{Estudo prático}

\begin{itemize}
  \item Implementação computacional das abordagens selecionadas.
  \item Validação e verificação da implementação computacional dos métodos de fusão e reconhecimento de padrões.
  \item Estudo dos dados computacional obtidos, realizando testes comparativos entre os métodos implementados.
\end{itemize}

\subsection{Resultado esperado}

\begin{itemize}
  \item Continuidade na identificação de técnicas acuradas de fusão propostas pelo estudo a partir da referência \cite{bmf_2020}, com a devida divulgação científica.
  \item Reconhecimento de padrões sem aplicação de filtros nos ruídos tipo \textit{speckle}.
\end{itemize}

\subsection{Divulgação dos resultado}

\begin{itemize}
  \item Elaboração de relatórios técnicos.
  \item Elaboração e submissão de artigos científicos para publicação em periódicos ou participações eventos, nacionais e internacionais.
\end{itemize}
\subsection{Cronograma}
O cronograma pode ser visto na figura (\ref{fig1}),
\begin{figure}[h!]
\begin{center}
\includegraphics[scale=0.8]{cronograma.jpg}
\end{center}
\caption{A figura representa o cronograma a ser seguido para a conclusão do projeto} \label{fig1}
\end{figure}


%
% ---- Bibliography ----
%
% BibTeX users should specify bibliography style 'splncs04'.
% References will then be sorted and formatted in the correct style.
%
% \bibliographystyle{splncs04}
% \bibliography{mybibliography}
%
\begin{thebibliography}{8}
% AAB inserido
\bibitem{bmf_2020}
A. A. De Borba, M. Marengoni, A. C. Frery, Artigo por aparecer em IEEE Geoscience and Remote Sensing Society Letters (GRSL).  DOI:10.1109/LGRS.2020.3022511. 

% AAB inserido
\bibitem{ztmxzxf}
X.~X. Zhu, D.~Tuia, L.~Mou, G.~Xia, L.~Zhang, F.~Xu, and F.~Fraundorfer, ``Deep
  learning in remote sensing: A comprehensive review and list of resources,''
  \emph{IEEE Geosci. Remote Sens. Mag.}, vol.~5, no.~4, pp. 8--36, Dec 2017.
% AAB inserido
\bibitem{nhfc}
A.~Nascimento, M.~Horta, A.~Frery, and R.~Cintra, ``Comparing edge detection
  methods based on stochastic entropies and distances for {P}ol{SAR} imagery,''
  \emph{IEEE J. Sel. Topics Appl. Earth Observ. Remote Sens.}, vol.~7, no.~2,
  pp. 648--663, 2014.

\bibitem{ref_proc1}
W. Chen, D. Hai, S. Gou and L. Jiao, "Classification of PolSAR Images Based on SVM with Self-Paced Learning Optimization," IGARSS 2018 - 2018 IEEE International Geoscience and Remote Sensing Symposium, Valencia, 2018, pp. 4491-4494, doi: 10.1109/IGARSS.2018.8517452.

\bibitem{ref_proc2}
Y. Duan, H. Duan and M. Sun, "Classification of the PolSAR Data Using Dual Classifiers," 2018 IEEE 3rd International Conference on Image, Vision and Computing (ICIVC), Chongqing, 2018, pp. 316-320, doi: 10.1109/ICIVC.2018.8492814.

\bibitem{ref_proc3}
H. Wang, F. Xu and Y. Jin, "A Review of Polsar Image Classification: from Polarimetry to Deep Learning," IGARSS 2019 - 2019 IEEE International Geoscience and Remote Sensing Symposium, Yokohama, Japan, 2019, pp. 3189-3192, doi: 10.1109/IGARSS.2019.8899902.

\bibitem{ref_proc4}
N. G. Kasapoglu, S. N. Anfinsen and T. Eltoft, "Fusion of optical and multifrequency polsar data for forest classification," 2012 IEEE International Geoscience and Remote Sensing Symposium, Munich, 2012, pp. 3355-3358, doi: 10.1109/IGARSS.2012.6350702.
\end{thebibliography}
\end{document}
