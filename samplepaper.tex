% This is samplepaper.tex, a sample chapter demonstrating the
% LLNCS macro package for Springer Computer Science proceedings;
% Version 2.20 of 2017/10/04
%
\documentclass[runningheads]{llncs}
%
\usepackage{graphicx}
\usepackage[brazil]{babel} % for European Portuguese use portuguese
\usepackage[fixlanguage]{babelbib}
\selectbiblanguage{brazil}

% Used for displaying a sample figure. If possible, figure files should
% be included in EPS format.
%
% If you use the hyperref package, please uncomment the following line
% to display URLs in blue roman font according to Springer's eBook style:
% \renewcommand\UrlFont{\color{blue}\rmfamily}

\begin{document}
%
\pagenumbering{gobble}
\title{Fusão de informações entre imagens PolSAR e ópticas para reconhecimento de padrões}
%
%\titlerunning{Abbreviated paper title}
% If the paper title is too long for the running head, you can set
% an abbreviated paper title here
%
\pagestyle{plain}
\author{Autor: Gabrielle Cibyl da Hora \\ 
Orientador: Mauricio Marengoni} 
%
%\authorrunning{F. Author et al.}
% First names are abbreviated in the running head.
% If there are more than two authors, 'et al.' is used.
%
\institute{PPGEEC - Programa de pós-graduação em Engenharia Elétrica e Computação. 
Universidade Presbiteriana Mackenzie.\\
\email{gcibyl@gmail.com}}
%
\maketitle              % typeset the header of the contribution
%
\begin{abstract}
% AAB modificado
O projeto de pesquisa propõe a fusão de informações para reconhecimento de padrões entre imagens ópticas e canais hh, hv e vv de imagens PolSAR múltiplas visadas. Os métodos de fusão foram escolhidos a partir das referências \cite{bmf_2020} e \cite{ref_proc3}.  Será estudado a viabilidade do métodos de fusão por média, transformada wavelet discreta de multi-resolução (MR-DWT), análise de componente principal (PCA), estatísticas ROC, transformada wavelet estacionária de multi-resolução (MR-SWT) e um método multi-resolução baseado na decomposição em valores singulares (MR-SVD) com a inserção das imagens ópticas. Com o objetivo de classificar de maneira acurada as bordas vamos propor duas técnica de reconhecimento de padrões. A primeira técnica baseada em votação majoritária e a segunda técnica baseada em propriedades estatísticas das imagens. 

\addto\captionsenglish{\renewcommand{\keywordsname}{Palavras-chave}}
\keywords{PolSAR  \and Métodos de fusão \and PCA \and SVM \and Inteligência Artificial}
\end{abstract}
%
%
%\
\section{Introdução}

O presente projeto de pesquisa é submetido ao Programa de Pós-graduação em Engenharia Elétrica e Computação da Universidade Presbiteriana Mackenzie. 

O radar de abertura sintética PolSAR alcançou uma posição essencial no sensoriamento remoto. Os dados que esses sensores fornecem precisam de sinais especificamente adaptados as técnicas de processamento. Entre essas técnicas, detecção de borda é uma das operações mais importantes para extrair informações. As bordas estão em um nível mais alto de abstração do que apenas dados e, como tal, fornecer informações relevantes sobre a imagem.



\section{Objetivos}

O principal objetivo do projeto é a aplicação das técnicas de reconhecimento de padrões para detecção de bordas sem utilizar métodos que retirem os ruídos. Este projeto de pesquisa também tem como intuito aplicar comparações entre técnicas de fusão, classificação e propõem dar continuidade no estudo realizado pelo Borba [1].

\section{Justificativas e motivações}
Este projeto de pesquisa tópicos atuais que servem de motivação ao projeto:
\begin{itemize}
  \item Aplicação em estudos florestais, pesquisa mineral e detecção de derramamento de óleo.
  \item O uso da polarimetria SAR em aplicações possui necessidade de estudos mais profundos para uma compreensão interação radar-alvo, disponibilidade de dados bem calibrados, algoritmos mais robustos, etc.
  \item Processamento de imagens, pois o Radar de abertura sintética polarimétrica (PolSAR) contém ruídos do tipo speckle.
\end{itemize}


\section{Conceitos}
% AAB inserido
O projeto pode ser dividido em uma primeira etapa que consiste em encontrar bordas nos diferentes canais (imagens) e a fusão dessas bordas como segunda etapa.
\subsection{Detecção de bordas em cada canal}
 Vamos modelar as imagens PolSAR com a função de densidade de probabilidade para os canais $(hh)$, $(hv)$ e $(vv)$ usando a distribuição Wishart (PDF) descrita por
\begin{equation}
    f_{\mathbf{Z}}(\mathbf{Z};\mathbf{\Sigma_{s}},L)=\frac{L^{mL}|\mathbf{Z}|^{L-m}}{|\mathbf{\Sigma_{s}}|^{L}\Gamma_m(L)} \exp(-L\mathbf{tr}(\mathbf{\Sigma_{s}}^{-1}\mathbf{Z})), \\
\end{equation} 
onde, $\mathbf{tr}(\cdot)$ é o operador traço de uma matriz, $\Gamma_m(L)$ é uma função Gamma multivariada definida por
\begin{equation}\label{cap_acf_10}
	\Gamma_m(L)=\pi^{\frac{1}{2}m(m-1)} \prod_{i=0}^{m-1}\Gamma(L-i) \\
\end{equation}
e $\Gamma(\cdot)$ é a função Gamma. Podemos afirmar que $\mathbf{Z}$ é distribuído como uma distribuição Wishart denotando por $\mathbf{Z}\sim W(\mathbf{\Sigma_{s}}, L)$ e satisfazendo $E[\mathbf{Z}]=\mathbf{\Sigma_{s}}$. Sem perda de generalidade para o texto vamos usar o simbolo $\mathbf{\Sigma}$ em detrimento a $\mathbf{\Sigma_{s}}$ para representar a matriz de covariância associada a $\mathbf{S}$.
 
Usando a função densidade de probabilidade e o método da máxima verossimilhança vamos detectar as bordas nas imagens PolSAR. Da mesma forma temos que definir o procedimento para imagens óticas. 
\subsection{Métodos de Fusão para detectar bordas em cada canal}
O objetivo da fusão de imagens é mesclar o conteúdo de várias imagens da mesma cena (canais diferentes) em uma única informação, assim, obtendo a melhor otimização ao comparar individualmente cada imagem. Os métodos de fusão a serem abordados serão:  transformada wavelet discreta de multi-resolução (MR-DWT), análise de componente principal (PCA), estatísticas ROC, transformada wavelet estacionária de multi-resolução (MR-SWT) e um método multi-resolução baseado na decomposição em valores singulares (MR-SVD).


\subsection{Provaveis métodos de classificação para fusão}
classificação é uma das questões mais fundamentais no
processo de interpretação de imagens PolSAR. Diversos estudos propõem desenvolvimentos de classificações de imagens PolSAR, tanto para métodos clássicos quanto híbridos. O projeto tem como intuito comparar e propor novas abordagens. Os métodos de classificação a serem abordados serão: SVM [2], deep Learning [3], forest Classification [4], entre outras heurísticas.



\section{Metodologia}

\subsection{Cronograma}
\begin{figure}
\begin{center}
\includegraphics[scale=0.8]{cronograma.jpg}
\end{center}
\caption{A figura representa o cronograma a ser seguido para a conclusão do projeto} \label{fig1}
\end{figure}


\subsection{Estudo teórico}
\begin{itemize}
  \item Realização da pesquisa bibliográfica e estudo dos artigos obtidos.
  \item Estudo da viabilidade e decisões sobre abordagens que serão propostas.
\end{itemize}

\subsection{Estudo prático}

\begin{itemize}
  \item Implementação computacional das abordagens selecionadas.
  \item Validação e verificação da implementação computacional dos métodos de fusão e reconhecimento de padrões.
  \item Estudo dos dados computacional obtidos, realizando testes comparativos entre os métodos implementados.
\end{itemize}

\subsection{Resultado esperado}

\begin{itemize}
  \item Continuidade na identificação de técnicas acuradas de fusão propostas pelo estudo a partir da referência [1], com artigos publicados em revistas.
  \item Reconhecimento de padrões sem aplicação de filtros nos ruídos.
\end{itemize}

\subsection{Divulgação dos resultado}

\begin{itemize}
  \item Elaboração de relatórios técnicos.
  \item Elaboração e submissão de artigos científicos para publicação em periódicos ou participações eventos, nacionais e internacionais.
\end{itemize}



%
% ---- Bibliography ----
%
% BibTeX users should specify bibliography style 'splncs04'.
% References will then be sorted and formatted in the correct style.
%
% \bibliographystyle{splncs04}
% \bibliography{mybibliography}
%
\begin{thebibliography}{8}
% AAB inserido
\bibitem{bmf_2020}
A. A. De Borba, M. Marengoni, A. C. Frery, Artigo por aparecer em IEEE Geoscience and Remote Sensing Society Letters (GRSL).  DOI:10.1109/LGRS.2020.3022511. 
\bibitem{ref_proc1}
W. Chen, D. Hai, S. Gou and L. Jiao, "Classification of PolSAR Images Based on SVM with Self-Paced Learning Optimization," IGARSS 2018 - 2018 IEEE International Geoscience and Remote Sensing Symposium, Valencia, 2018, pp. 4491-4494, doi: 10.1109/IGARSS.2018.8517452.

\bibitem{ref_proc1}
Y. Duan, H. Duan and M. Sun, "Classification of the PolSAR Data Using Dual Classifiers," 2018 IEEE 3rd International Conference on Image, Vision and Computing (ICIVC), Chongqing, 2018, pp. 316-320, doi: 10.1109/ICIVC.2018.8492814.

\bibitem{ref_proc2}
H. Wang, F. Xu and Y. Jin, "A Review of Polsar Image Classification: from Polarimetry to Deep Learning," IGARSS 2019 - 2019 IEEE International Geoscience and Remote Sensing Symposium, Yokohama, Japan, 2019, pp. 3189-3192, doi: 10.1109/IGARSS.2019.8899902.

\bibitem{ref_proc3}
N. G. Kasapoglu, S. N. Anfinsen and T. Eltoft, "Fusion of optical and multifrequency polsar data for forest classification," 2012 IEEE International Geoscience and Remote Sensing Symposium, Munich, 2012, pp. 3355-3358, doi: 10.1109/IGARSS.2012.6350702.
\end{thebibliography}
\end{document}
